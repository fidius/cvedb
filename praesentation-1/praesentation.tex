
\documentclass[authorinfoot]{fidius-p}
% \usepackage{tabularx}
\author{AB \and JF \and LM \and DK}
% \title[short title]{long title}
% optional: \subtitle[short subtitle]{long subtitle}
\title{OpenVAS \& CVE-DB}
\subtitle{Zwischenpräsentation}
\date{10.12.2010}

\setcounter{tocdepth}{1}

\setbeamertemplate{bibliography item}[text]

\setbeamertemplate{bibliography entry title}{}
\setbeamertemplate{bibliography entry location}{}
\setbeamertemplate{bibliography entry note}{}

\bibliographystyle{is-plain}

\begin{document}

\frame{\titlepage}
\frame{\tableofcontents}

\section{OpenVAS}
\secframe{Ziele}{
  \begin{itemize}
    \item Wir wollen evaluieren, ob sich OpenVAS zur Ermittlung von CVE-Nummern
auf Basis zuvor gefundener Schwachstellen verwenden lässt.
    \subitem{Dadurch Reduzierung der verwendeten Exploits.}
  \end{itemize}
}

\secframe{Ergebnisse}{
  \begin{itemize}
    \item OpenVAS Architektur aufwendig zu installieren
    \subitem{Scanner, Manager, Client, Benutzer und Zertifikate}
    \item libopenvas schlecht dokumentiert.
    \item Viele Schwachstellentests nicht für remote-Exploits geeignet.
    \item Scan-Ergebnisse enthalten selten CVE-Nummern
    \subitem{Und diese entsprechen nicht immer den in Metasploit verwendeten}
  \end{itemize}
}

\secframe{Retrospektive}{ % TODO mehr sinnvolles
  \begin{itemize}
    \item Regelmäßige Gruppentreffen (wie immer) sinnvoll.
    \item Normales Beleidigungsniveau (wie immer).
  \end{itemize}
}

\section{CVE-DB}
\secframe{Ziele}{
  \begin{itemize}
    \item Abfrage von CVE-Nummern anhand vorgefundener Software
    \item Bereitstellen der Daten für andere FIDIUS Komponenten
  \end{itemize}
}

\secframe{Ergebnisse}{
  \begin{itemize}
    \item Verwendung der NVD-Datenbank
    \item Einlesen von XML-Dateien, die CVE-Einträge beinhalten
    \item Datenbankschema für eine Rails-Anwendung
    \item Speichern der CVE-Einträge in Rails-DB
    \subitem{Abfrage über Active Record}
    \item Rake-Tasks um zu parsen, neue XML-Dateien abzurufen, etc.
    \subitem{\texttt{nvd:parse[<datei>], nvd:get[<datei>], nvd:list\_local,
nvd:list\_remote, nvd:update} (eckige Klammern gehören dazu)}
  \end{itemize}
}

\secframe{Probleme}{
  \begin{itemize}
    \item Sehr große Datenmengen die gespeichert werden müssen
    \item Gefahr der Redundanz bei CVE-Einträgen die aktualisiert werden
    \item Überprüfung auf Redundanz dauert sehr lange
    \subitem{ca. 4 1/2 Stunden pro 30MB-XML wenn alle Attribute auf Redundanzen
überprüft werden.}
    \subitem{Unser Lösungsansatz: Erst ohne Überprüfung speichern, später
Redundanzen beseitigen.}
  \end{itemize}
}

\secframe{Retrospektive}{
  \begin{itemize}
    \item Pair-Programming sinnvoll
    \item Einstieg in Rails dadurch einfacher
    \item Freitags gemeinsam arbeiten sinnvoll, dadurch kurze
Kommunikationswege zwischen den Gruppen
  \end{itemize}
}

\end{document}

